% ~~~~~~~~~~~~~~~~~~~~~~~~~~~~~~~~~~~~~~~~~~~~~~~~~~~~~~~~~~~~~~~~~~~~~~ %
% ~~~~~~~~~~~~~~~~~~~~~~ Erweiterung des Präambel ~~~~~~~~~~~~~~~~~~~~~~ %
% ~~~~~~~~~~~~~~~~~~~~~~~~~~~~~~~~~~~~~~~~~~~~~~~~~~~~~~~~~~~~~~~~~~~~~~ %

% Shortcuts für häufig wiederkehrende Begriffe
\newcommand{\asw}{ASW}

% Wenn nach den 3 Ebenen nach \subsubsection eine weitere Gliederungsebene benötigt wird
\newcommand{\paragraphheader}[1]{\paragraph{#1}\mbox{}\\}

% Vordefinierte Arten der Fußnoten. 
% Die Struktur kann an dieser Stelle geändert werden und betrifft jede Verwendung im gesamten Dokument,
% was praktisch sein kann, falls der Betreuer oder Gutachter zb. die ausgeschriebene Variante von "Vgl." bevorzugt.
\newcommand{\vgl}[2]{\footcite[Vgl.][#2]{#1}}
\newcommand{\footrefnote}[2]{\footnote{Für das Thema #1 siehe Kapitel~\ref{#2}}}
\newcommand{\wholesection}[2]{\footcite[Für den gesamten Abschnitt vgl.][#2]{#1}}

% Zur Generierung von Blindtext - kann entfernt werden
\usepackage{lipsum}

% ~~~~~~~~~~~~~~~~~~~~~~~~~~~~~~~~~~~~~~~~~~~~~~~~~~~~~~~~~~~~~~~~~~~~~~ %
% ~~~~~~~~~~~~~~~~~~~ Informationen über die Arbeit ~~~~~~~~~~~~~~~~~~~~ %
% ~~~~~~~~~~~~~~~~~~~~~~~~~~~~~~~~~~~~~~~~~~~~~~~~~~~~~~~~~~~~~~~~~~~~~~ %

\title{\LaTeX-Template für wissenschaftliche Arbeiten an der \asw}
\author{Max Mustermann}
\date{\today}

% Weitere Variablen innerhalb von LaTeX für das Titelblatt
\newcommand{\varMartrikelnummer}{XXXXXXX}
\newcommand{\varArbeit}{Bachelorarbeit}
\newcommand{\varStudiengang}{Wirtschaftsinformatik}
\newcommand{\varUnternehmen}{Dummy Company}
\newcommand{\varBetrBetreuer}{Erika Musterfrau}
\newcommand{\varASWGutachter}{Prof. Dr. Dieter Hofbauer}
\newcommand{\varEingereichtAm}{01. Januar 2020}


% ~~~~~~~~~~~~~~~~~~~~~~~~~~~~~~~~~~~~~~~~~~~~~~~~~~~~~~~~~~~~~~~~~~~~~~ %
% ~~~~~~~~~~~~~~~~~~~~~~~~~~~~ Customizing ~~~~~~~~~~~~~~~~~~~~~~~~~~~~~ %
% ~~~~~~~~~~~~~~~~~~~~~~~~~~~~~~~~~~~~~~~~~~~~~~~~~~~~~~~~~~~~~~~~~~~~~~ %

% Titelblatt: environment/titlepage | environment/titlepage_modern
\newcommand{\varTitlepage}{environment/titlepage}

% Der Name der Datei für das Firmenlogo. Falls keins vorhanden ist wird es aber keinen Fehler geben.
% Die Datei muss im Verzeichnis der Bilder liegen und wird ohne Endung hier angegeben.
\newcommand{\varCompanyLogoFile}{logo_company}

% Wenn environment/titlepage gewählt und \varCompanyLogoFile gesetzt ist
% muss folgende Angabe an das Logo angepasst werden um die Bilder in die vertikale Mitte des Banners zu setzen
% Der passende Wert kann von 25pt abweichen (am besten per trial-and-error herausfinden)
%\newcommand{\varTitlepageLogoMarginTop}{25pt}

% Schriftart: Times New Roman
\renewcommand{\rmdefault}{ptm}

% Ob in den Verzeichnissen eine deklarative Abkürzung stehen soll (zB. "Abb. 1")
% Wenn es nicht gewollt ist muss die nächste Zeile auskommentiert werden.
\newcommand{\varShowTitlesInLists}{true}

% Klammern um die Jahresangabe in Fußnoten und im Literaturverzeichnis ausblenden
% Wenn es nicht gewollt ist muss die nächste Zeile auskommentiert werden.
%\renewcommand{\varNoParenthesesAroundYear}{true}

% Ob noch vor dem Inhaltsverzeichnis ein Sperrvermerk angezeigt werden soll.
% Der Sperrvermerk kann in environment/sperrvermerk.tex angepasst werden.
% Wenn es nicht gewollt ist muss die nächste Zeile auskommentiert werden.
\newcommand{\varShowBlockingNote}{true}


% ~~~~~~~~~~~~~~~~~~~~~~~~~~~~~~~~~~~~~~~~~~~~~~~~~~~~~~~~~~~~~~~~~~~~~~ %
% ~~~~~~~~~~~~~~~~~~~~~~~~~~~ Silbentrennung ~~~~~~~~~~~~~~~~~~~~~~~~~~~ %
% ~~~~~~~~~~~~~~~~~~~~~~~~~~~~~~~~~~~~~~~~~~~~~~~~~~~~~~~~~~~~~~~~~~~~~~ %

% Falls LaTeX Wörter nicht richtig trennt können hier eigene Angaben in folgendem Stil gemacht werden
\hyphenation{Java-Script}
