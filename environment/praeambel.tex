\usepackage[utf8]{inputenc}

\usepackage[rgb]{xcolor}
\usepackage{embedfile}

% Zur Anpassung der Seitenränder
\usepackage[a4paper,left=3cm,right=2cm,top=3cm,bottom=2.5cm]{geometry}

% Zeilenabstände anpassen
\usepackage{setspace}

% Deutsch: https://de.overleaf.com/learn/German
\usepackage[T1]{fontenc}
\usepackage[ngerman]{babel}

% Anführungszeichen unten: "`
% Anführungszeichen oben: "'
\usepackage{csquotes}

% um Grafiken einzubinden, {Angabe des Pfades der Bilder}
\usepackage{graphicx}
\graphicspath{ {images/} }

% zur korrekten Platzierung der Banner im Titelblatt
\usepackage{chngpage}
\usepackage{calc}
\usepackage{float}

% Automatisches Generieren von Hyperlinks bei Verweisen und URLs
\usepackage{url}
\usepackage[pdfa]{hyperref}
\usepackage{hyperxmp}

% sodass LoT, LoF und Literatur in ToC erscheint
\usepackage[nottoc,numbib]{tocbibind}

% Listingverzeichnis
\usepackage{listings}
\renewcommand{\lstlistlistingname}{Listingverzeichnis}

% LoT und LoF nur ausgeben, wenn es Einträge gibt
% Mit diesem Packet werden alle vorhandenen Grafiken, Tabellen und Listings gezählt
\usepackage[figure,table,lstlisting]{totalcount}

% Akronyme mit Einstellungen bezüglich Gestaltung des Verzeichnisses
\usepackage[acronym, nogroupskip, nonumberlist, nopostdot]{glossaries}
\loadglsentries{environment/acronym}
\makenoidxglossaries
\setacronymstyle{long-sc-short}

% Einstellen von environments und captions
\usepackage[hang]{footmisc}
\usepackage{wrapfig}
\usepackage[font=small, justification=centering]{caption}
\usepackage{booktabs}
\usepackage{enumitem}
\setlist[itemize]{itemsep=0cm}

% Setzen von Abständen innerhalt von Fußnoten: für mehr Möglichkeiten siehe https://latex.org/forum/viewtopic.php?t=6781
\renewcommand{\footnotemargin}{12pt}

% Biblatex als Literaturverwaltung
% Siehe auch die Datei biblatex.cfg
% Dokumentation: https://ctan.kako-dev.de/macros/latex/contrib/biblatex/doc/biblatex.pdf
\usepackage[style=ext-authoryear,     % ext- ermöglicht das Einblenden der Klammern um die Jahreszahl in Fußnoten
            sorting=nyt,              % Nach Nachnamen des ersten genannten Autorens sortieren, dann Jahr, dann Titel
            labeldate=year,           % Nur Jahreszahl im Label
            mergedate=basic,          % Datum hinten nur, wenn genauer als Jahreszahl
            isbn=false,               % Ausblenden des Feldes
            url=false,                % Ausblenden des Feldes
            doi=true,                 % Einblenden des Feldes
            eprint=true,              % Einblenden des Feldes
            clearlang=true,           % Ausblenden von Sprachcodes
            maxcitenames=2,           % Ab drei Autoren mit "et at." abkürzen
            citexref=true,            % Aufsplitten von Referenz in Sammelwerken
            mincrossrefs=0,           % Aufsplitten direkt bei der ersten Referenz auf Sammelwerk
            backref=true,             % Rückreferenzen vom Literaturverzeichnis in den Text
            maxbibnames=100]{biblatex}% Alle Autoren im Literaturverzeichnis ausschreiben

\DefineBibliographyStrings{german}{
  backrefpage={zitiert auf S\addperiod},
  backrefpages={zitiert auf S\addperiod}
}

% Einbinden der Literatureinträge
\addbibresource{references.bib}

% Zeilenumbrüche in Url bei jedem beliebigen Buchstaben,
% um overfull H-Boxes im Literaturverzeichnis zu vermeiden
\setcounter{biburllcpenalty}{1000}

% ~~~~~~~~~~~~~~~~~~~~~~~~~~~~~~~~~~~~~~~~~~~~~~~~~~~~~~~~~~~~~~~~~~~~~~ %
% ~~~~~~~~~~~~~~~~~~~~~~~~~~~ Konfiguration ~~~~~~~~~~~~~~~~~~~~~~~~~~~~ %
% ~~~~~~~~~~~~~~~~~~~~~~~~~~~~~~~~~~~~~~~~~~~~~~~~~~~~~~~~~~~~~~~~~~~~~~ %

% Projektspezifische Einstellungen laden
% In dieser Datei müssen lediglich die vorgegebenen Variablen geändert werden
% ~~~~~~~~~~~~~~~~~~~~~~~~~~~~~~~~~~~~~~~~~~~~~~~~~~~~~~~~~~~~~~~~~~~~~~ %
% ~~~~~~~~~~~~~~~~~~~~~~ Erweiterung des Präambel ~~~~~~~~~~~~~~~~~~~~~~ %
% ~~~~~~~~~~~~~~~~~~~~~~~~~~~~~~~~~~~~~~~~~~~~~~~~~~~~~~~~~~~~~~~~~~~~~~ %

% Shortcuts für häufig wiederkehrende Begriffe
\newcommand{\asw}{ASW}

% Wenn nach den 3 Ebenen nach \subsubsection eine weitere Gliederungsebene benötigt wird
\newcommand{\paragraphheader}[1]{\paragraph{#1}\mbox{}\\}

% Vordefinierte Arten der Fußnoten. 
% Die Struktur kann an dieser Stelle geändert werden und betrifft jede Verwendung im gesamten Dokument,
% was praktisch sein kann, falls der Betreuer oder Gutachter zb. die ausgeschriebene Variante von "Vgl." bevorzugt.
\newcommand{\vgl}[2]{\footcite[Vgl.][#2]{#1}}
\newcommand{\footrefnote}[2]{\footnote{Für das Thema #1 siehe Kapitel~\ref{#2}}}
\newcommand{\wholesection}[2]{\footcite[Für den gesamten Abschnitt vgl.][#2]{#1}}

% Zur Generierung von Blindtext - kann entfernt werden
\usepackage{lipsum}

% ~~~~~~~~~~~~~~~~~~~~~~~~~~~~~~~~~~~~~~~~~~~~~~~~~~~~~~~~~~~~~~~~~~~~~~ %
% ~~~~~~~~~~~~~~~~~~~ Informationen über die Arbeit ~~~~~~~~~~~~~~~~~~~~ %
% ~~~~~~~~~~~~~~~~~~~~~~~~~~~~~~~~~~~~~~~~~~~~~~~~~~~~~~~~~~~~~~~~~~~~~~ %

\title{\LaTeX-Template für wissenschaftliche Arbeiten an der \asw}
\author{Max Mustermann}
\date{\today}

% Weitere Variablen innerhalb von LaTeX für das Titelblatt
\newcommand{\varMartrikelnummer}{XXXXXXX}
\newcommand{\varArbeit}{Bachelorarbeit}
\newcommand{\varStudiengang}{Wirtschaftsinformatik}
\newcommand{\varUnternehmen}{Dummy Company}
\newcommand{\varBetrBetreuer}{Erika Musterfrau}
\newcommand{\varASWGutachter}{Prof. Dr. Dieter Hofbauer}
\newcommand{\varEingereichtAm}{01. Januar 2020}


% ~~~~~~~~~~~~~~~~~~~~~~~~~~~~~~~~~~~~~~~~~~~~~~~~~~~~~~~~~~~~~~~~~~~~~~ %
% ~~~~~~~~~~~~~~~~~~~~~~~~~~~~ Customizing ~~~~~~~~~~~~~~~~~~~~~~~~~~~~~ %
% ~~~~~~~~~~~~~~~~~~~~~~~~~~~~~~~~~~~~~~~~~~~~~~~~~~~~~~~~~~~~~~~~~~~~~~ %

% Titelblatt: environment/titlepage | environment/titlepage_modern
\newcommand{\varTitlepage}{environment/titlepage}

% Der Name der Datei für das Firmenlogo. Falls keins vorhanden ist wird es aber keinen Fehler geben.
% Die Datei muss im Verzeichnis der Bilder liegen und wird ohne Endung hier angegeben.
\newcommand{\varCompanyLogoFile}{logo_company}

% Wenn environment/titlepage gewählt und \varCompanyLogoFile gesetzt ist
% muss folgende Angabe an das Logo angepasst werden um die Bilder in die vertikale Mitte des Banners zu setzen
% Der passende Wert kann von 25pt abweichen (am besten per trial-and-error herausfinden)
%\newcommand{\varTitlepageLogoMarginTop}{25pt}

% Schriftart: Times New Roman
\renewcommand{\rmdefault}{ptm}

% Ob in den Verzeichnissen eine deklarative Abkürzung stehen soll (zB. "Abb. 1")
% Wenn es nicht gewollt ist muss die nächste Zeile auskommentiert werden.
\newcommand{\varShowTitlesInLists}{true}

% Klammern um die Jahresangabe in Fußnoten und im Literaturverzeichnis ausblenden
% Wenn es nicht gewollt ist muss die nächste Zeile auskommentiert werden.
%\renewcommand{\varNoParenthesesAroundYear}{true}

% Ob noch vor dem Inhaltsverzeichnis ein Sperrvermerk angezeigt werden soll.
% Der Sperrvermerk kann in environment/sperrvermerk.tex angepasst werden.
% Wenn es nicht gewollt ist muss die nächste Zeile auskommentiert werden.
\newcommand{\varShowBlockingNote}{true}

% Gemäß Leitfaden der ASW soll nur in Bachelorarbeiten ein Seitenumbruch vor
% einem neuen Abschnitt gemacht werden.
\newcommand{\bachelorarbeit}{Bachelorarbeit}
\newcommand{\conditionalPageBreak}{%
  \ifdefined\varArbeit
    \ifx\varArbeit\bachelorarbeit
      \clearpage%
    \fi
  \fi
}

% ~~~~~~~~~~~~~~~~~~~~~~~~~~~~~~~~~~~~~~~~~~~~~~~~~~~~~~~~~~~~~~~~~~~~~~ %
% ~~~~~~~~~~~~~~~~~~~~~~~~~~~ Silbentrennung ~~~~~~~~~~~~~~~~~~~~~~~~~~~ %
% ~~~~~~~~~~~~~~~~~~~~~~~~~~~~~~~~~~~~~~~~~~~~~~~~~~~~~~~~~~~~~~~~~~~~~~ %

% Falls LaTeX Wörter nicht richtig trennt können hier eigene Angaben in folgendem Stil gemacht werden
\hyphenation{Java-Script}


% Ob "Abb.", "Tbl." und "Lst." vor den Nummern der Verzeichnisse erscheinen
\ifdefined\varShowTitlesInLists
  \makeatletter
  \renewcommand{\l@figure}[2]{\@dottedtocline{1}{1.5em}{2.3em}{Abb. #1}{#2}}
  \renewcommand{\l@table}[2]{\@dottedtocline{1}{1.5em}{2.3em}{Tbl. #1}{#2}}
  \renewcommand{\l@lstlisting}[2]{\@dottedtocline{1}{1.5em}{2.3em}{Lst. #1}{#2}}
  \makeatother
\fi

% Ausblenden der Klammern um die Jahresangabe im Literaturverzeichnis
\ifdefined\varNoParenthesesAroundYear
  \makeatletter
  \def\act@on@bibmacro#1#2{%
    \expandafter#1\csname abx@macro@\detokenize{#2}\endcsname
  }
  \def\patchbibmacro{\act@on@bibmacro\patchcmd}
  \def\pretobibmacro{\act@on@bibmacro\pretocmd}
  \def\apptobibmacro{\act@on@bibmacro\apptocmd}
  \def\showbibmacro{\act@on@bibmacro\show}
  \makeatother

  \patchbibmacro{date+extradate}{%
  \printtext[parens]%
  }{%
  \setunit{\addperiod\space}%
  \printtext%
  }{}{}
\fi

% Include RGB color profile for PDF/A conformance
\immediate\pdfobj stream attr{/N 3^^J/Alternate/DeviceRGB} file{sRGB.icc}
\pdfcatalog{
  /PageMode /UseNone
  /OutputIntents [
    <<
      /Type /OutputIntent
      /S /GTS_PDFA1
      /DestOutputProfile \the\pdflastobj\space 0 R
      /OutputConditionIdentifier (sRGB IEC61966-2.1 (Equivalent to www.srgb.com 1998 HP profile))
      /Info(sRGB IEC61966-2.1 (Equivalent to www.srgb.com 1998 HP profile))
      /RegistryName (http://www.color.org/)
    >>
  ]
}%

\hypersetup{
  colorlinks      = true,     % Colours links instead of ugly boxes
  urlcolor        = black,    % Colour for external hyperlinks
  linkcolor       = black,    % Colour of internal links
  citecolor       = black,    % Colour of citations
  pdfapart        = 3,        % PDF/A version
  pdfaconformance = B         % PDF/A conformance level (B=basic)
}

\setlength{\parskip}{1.5ex plus0.5ex minus0.3ex}
\setlength{\parindent}{0ex} 
