% ~~~~~~~~~~~~~~~~~~~~~~~~~~~~~~~~~~~~~~~~~~~~~~~~~~~~~~~~~~~~~~~~~~~~~~ %
% ~~~~~~~~~~~~~~~~~~~~~~~~~~~~~~ License ~~~~~~~~~~~~~~~~~~~~~~~~~~~~~~~ %
% ~~~~~~~~~~~~~~~~~~~~~~~~~~~~~~~~~~~~~~~~~~~~~~~~~~~~~~~~~~~~~~~~~~~~~~ %

% CC BY-NC-SA 3.0 (http://creativecommons.org/licenses/by-nc-sa/3.0/)
% Siehe auch LICENCE.txt


% ~~~~~~~~~~~~~~~~~~~~~~~~~~~~~~~~~~~~~~~~~~~~~~~~~~~~~~~~~~~~~~~~~~~~~~ %
% ~~~~~~~~~~~~~~~~~~~~~~~~~~~~~~ Präambel ~~~~~~~~~~~~~~~~~~~~~~~~~~~~~~ %
% ~~~~~~~~~~~~~~~~~~~~~~~~~~~~~~~~~~~~~~~~~~~~~~~~~~~~~~~~~~~~~~~~~~~~~~ %

% Artikel-Klasse mit a4paper Option für die Seitenränder
\documentclass[12pt,a4paper]{article}
\usepackage[utf8]{inputenc}

% Zur Anpassung der Seitenränder
\usepackage[a4paper,left=3cm,right=2cm,top=3cm,bottom=2.5cm]{geometry}

% Zeilenabstände anpassen
\usepackage{setspace}

% Deutsch: https://de.overleaf.com/learn/German
\usepackage[T1]{fontenc}
\usepackage[ngerman]{babel}

% Anführungszeichen unten: "`
% Anführungszeichen oben: "'
\usepackage{csquotes}

% um Grafiken einzubinden, {Angabe des Pfades der Bilder}
\usepackage{graphicx}
\graphicspath{ {images/} }

% zur korrekten Platzierung der Banner im Titelblatt
\usepackage{chngpage}
\usepackage{calc}
\usepackage{float}

% Automatisches Generieren von Hyperlinks bei Verweisen und URLs
\usepackage{url}
\usepackage{hyperref}
\hypersetup{
  colorlinks   = true,     % Colours links instead of ugly boxes
  urlcolor     = black,    % Colour for external hyperlinks
  linkcolor    = black,    % Colour of internal links
  citecolor    = black     % Colour of citations
}

% sodass LoT, LoF und Literatur in ToC erscheint
\usepackage[nottoc,numbib]{tocbibind}

% Listingverzeichnis
\usepackage{listings}
\renewcommand{\lstlistlistingname}{Listingverzeichnis}

% LoT und LoF nur ausgeben, wenn es Einträge gibt
% Mit diesem Packet werden alle vorhandenen Grafiken, Tabellen und Listings gezählt
\usepackage[figure,table,lstlisting]{totalcount}

% Akronyme mit Einstellungen bezüglich Gestaltung des Verzeichnisses
\usepackage[acronym, nogroupskip, nonumberlist, nopostdot]{glossaries}
\loadglsentries{environment/acronym}
\makenoidxglossaries
\setacronymstyle{long-sc-short}

% Einstellen von environments und captions
\usepackage[hang]{footmisc}
\usepackage{wrapfig}
\usepackage[font=small, justification=centering]{caption}
\usepackage{booktabs}
\usepackage{enumitem}
\setlist[itemize]{itemsep=0cm}

% Setzen von Abständen innerhalt von Fußnoten: für mehr Möglichkeiten siehe https://latex.org/forum/viewtopic.php?t=6781
\renewcommand{\footnotemargin}{12pt}

% Biblatex als Literaturverwaltung
% Siehe auch die Datei biblatex.cfg
% Dokumentation: https://ctan.kako-dev.de/macros/latex/contrib/biblatex/doc/biblatex.pdf
\usepackage[style=ext-authoryear,     % ext- ermöglicht das Einblenden der Klammern um die Jahreszahl in Fußnoten
            sorting=nyt,              % Nach Nachnamen des ersten genannten Autorens sortieren, dann Jahr, dann Titel
            isbn=false,               % Ausblenden des Feldes
            url=false,                % Ausblenden des Feldes
            doi=true,                 % Einblenden des Feldes
            eprint=false,             % Ausblenden des Feldes
            maxcitenames=2,           % Ab drei Autoren mit "et at." abkürzen
            maxbibnames=100]{biblatex}% Alle Autoren im Literaturverzeichnis ausschreiben

% Einbinden der Literatureinträge
\addbibresource{references.bib}

% Zeilenumbrüche in Url bei jedem beliebigen Buchstaben,
% um overfull H-Boxes im Literaturverzeichnis zu vermeiden
\setcounter{biburllcpenalty}{1000}


% ~~~~~~~~~~~~~~~~~~~~~~~~~~~~~~~~~~~~~~~~~~~~~~~~~~~~~~~~~~~~~~~~~~~~~~ %
% ~~~~~~~~~~~~~~~~~~~~~~~~~~~ Konfiguration ~~~~~~~~~~~~~~~~~~~~~~~~~~~~ %
% ~~~~~~~~~~~~~~~~~~~~~~~~~~~~~~~~~~~~~~~~~~~~~~~~~~~~~~~~~~~~~~~~~~~~~~ %

% Projektspezifische Einstellungen laden
% In dieser Datei müssen lediglich die vorgegebenen Variablen geändert werden
% ~~~~~~~~~~~~~~~~~~~~~~~~~~~~~~~~~~~~~~~~~~~~~~~~~~~~~~~~~~~~~~~~~~~~~~ %
% ~~~~~~~~~~~~~~~~~~~~~~ Erweiterung des Präambel ~~~~~~~~~~~~~~~~~~~~~~ %
% ~~~~~~~~~~~~~~~~~~~~~~~~~~~~~~~~~~~~~~~~~~~~~~~~~~~~~~~~~~~~~~~~~~~~~~ %

% Shortcuts für häufig wiederkehrende Begriffe
\newcommand{\asw}{ASW}

% Wenn nach den 3 Ebenen nach \subsubsection eine weitere Gliederungsebene benötigt wird
\newcommand{\paragraphheader}[1]{\paragraph{#1}\mbox{}\\}

% Vordefinierte Arten der Fußnoten. 
% Die Struktur kann an dieser Stelle geändert werden und betrifft jede Verwendung im gesamten Dokument,
% was praktisch sein kann, falls der Betreuer oder Gutachter zb. die ausgeschriebene Variante von "Vgl." bevorzugt.
\newcommand{\vgl}[2]{\footcite[Vgl.][#2]{#1}}
\newcommand{\footrefnote}[2]{\footnote{Für das Thema #1 siehe Kapitel~\ref{#2}}}
\newcommand{\wholesection}[2]{\footcite[Für den gesamten Abschnitt vgl.][#2]{#1}}

% Zur Generierung von Blindtext - kann entfernt werden
\usepackage{lipsum}

% ~~~~~~~~~~~~~~~~~~~~~~~~~~~~~~~~~~~~~~~~~~~~~~~~~~~~~~~~~~~~~~~~~~~~~~ %
% ~~~~~~~~~~~~~~~~~~~ Informationen über die Arbeit ~~~~~~~~~~~~~~~~~~~~ %
% ~~~~~~~~~~~~~~~~~~~~~~~~~~~~~~~~~~~~~~~~~~~~~~~~~~~~~~~~~~~~~~~~~~~~~~ %

\title{\LaTeX-Template für wissenschaftliche Arbeiten an der \asw}
\author{Max Mustermann}
\date{\today}

% Weitere Variablen innerhalb von LaTeX für das Titelblatt
\newcommand{\varMartrikelnummer}{XXXXXXX}
\newcommand{\varArbeit}{Bachelorarbeit}
\newcommand{\varStudiengang}{Wirtschaftsinformatik}
\newcommand{\varUnternehmen}{Dummy Company}
\newcommand{\varBetrBetreuer}{Erika Musterfrau}
\newcommand{\varASWGutachter}{Prof. Dr. Dieter Hofbauer}
\newcommand{\varEingereichtAm}{01. Januar 2020}


% ~~~~~~~~~~~~~~~~~~~~~~~~~~~~~~~~~~~~~~~~~~~~~~~~~~~~~~~~~~~~~~~~~~~~~~ %
% ~~~~~~~~~~~~~~~~~~~~~~~~~~~~ Customizing ~~~~~~~~~~~~~~~~~~~~~~~~~~~~~ %
% ~~~~~~~~~~~~~~~~~~~~~~~~~~~~~~~~~~~~~~~~~~~~~~~~~~~~~~~~~~~~~~~~~~~~~~ %

% Titelblatt: environment/titlepage | environment/titlepage_modern
\newcommand{\varTitlepage}{environment/titlepage}

% Der Name der Datei für das Firmenlogo. Falls keins vorhanden ist wird es aber keinen Fehler geben.
% Die Datei muss im Verzeichnis der Bilder liegen und wird ohne Endung hier angegeben.
\newcommand{\varCompanyLogoFile}{logo_company}

% Wenn environment/titlepage gewählt und \varCompanyLogoFile gesetzt ist
% muss folgende Angabe an das Logo angepasst werden um die Bilder in die vertikale Mitte des Banners zu setzen
% Der passende Wert kann von 25pt abweichen (am besten per trial-and-error herausfinden)
%\newcommand{\varTitlepageLogoMarginTop}{25pt}

% Schriftart: Times New Roman
\renewcommand{\rmdefault}{ptm}

% Ob in den Verzeichnissen eine deklarative Abkürzung stehen soll (zB. "Abb. 1")
% Wenn es nicht gewollt ist muss die nächste Zeile auskommentiert werden.
\newcommand{\varShowTitlesInLists}{true}

% Klammern um die Jahresangabe in Fußnoten und im Literaturverzeichnis ausblenden
% Wenn es nicht gewollt ist muss die nächste Zeile auskommentiert werden.
%\renewcommand{\varNoParenthesesAroundYear}{true}

% Ob noch vor dem Inhaltsverzeichnis ein Sperrvermerk angezeigt werden soll.
% Der Sperrvermerk kann in environment/sperrvermerk.tex angepasst werden.
% Wenn es nicht gewollt ist muss die nächste Zeile auskommentiert werden.
\newcommand{\varShowBlockingNote}{true}


% ~~~~~~~~~~~~~~~~~~~~~~~~~~~~~~~~~~~~~~~~~~~~~~~~~~~~~~~~~~~~~~~~~~~~~~ %
% ~~~~~~~~~~~~~~~~~~~~~~~~~~~ Silbentrennung ~~~~~~~~~~~~~~~~~~~~~~~~~~~ %
% ~~~~~~~~~~~~~~~~~~~~~~~~~~~~~~~~~~~~~~~~~~~~~~~~~~~~~~~~~~~~~~~~~~~~~~ %

% Falls LaTeX Wörter nicht richtig trennt können hier eigene Angaben in folgendem Stil gemacht werden
\hyphenation{Java-Script}


% Ob "Abb.", "Tbl." und "Lst." vor den Nummern der Verzeichnisse erscheinen
\ifdefined\varShowTitlesInLists
  \makeatletter
  \renewcommand{\l@figure}[2]{\@dottedtocline{1}{1.5em}{2.3em}{Abb. #1}{#2}}
  \renewcommand{\l@table}[2]{\@dottedtocline{1}{1.5em}{2.3em}{Tbl. #1}{#2}}
  \renewcommand{\l@lstlisting}[2]{\@dottedtocline{1}{1.5em}{2.3em}{Lst. #1}{#2}}
  \makeatother
\fi

% Ausblenden der Klammern um die Jahresangabe im Literaturverzeichnis
\ifdefined\varNoParenthesesAroundYear
  \makeatletter
  \def\act@on@bibmacro#1#2{%
    \expandafter#1\csname abx@macro@\detokenize{#2}\endcsname
  }
  \def\patchbibmacro{\act@on@bibmacro\patchcmd}
  \def\pretobibmacro{\act@on@bibmacro\pretocmd}
  \def\apptobibmacro{\act@on@bibmacro\apptocmd}
  \def\showbibmacro{\act@on@bibmacro\show}
  \makeatother

  \patchbibmacro{date+extradate}{%
  \printtext[parens]%
  }{%
  \setunit{\addperiod\space}%
  \printtext%
  }{}{}
\fi


% ~~~~~~~~~~~~~~~~~~~~~~~~~~~~~~~~~~~~~~~~~~~~~~~~~~~~~~~~~~~~~~~~~~~~~~ %
% ~~~~~~~~~~~~~~~~~~~~~~~~~~~~~~ Dokument ~~~~~~~~~~~~~~~~~~~~~~~~~~~~~~ %
% ~~~~~~~~~~~~~~~~~~~~~~~~~~~~~~~~~~~~~~~~~~~~~~~~~~~~~~~~~~~~~~~~~~~~~~ %

\begin{document}

  % Titelblatt
  \input{\varTitlepage}
  \newpage

  \ifdefined\varShowBlockingNote
    \setstretch{1.2} 
    \section*{Sperrvermerk}
Die vorliegende \varArbeit \space basiert auf internen, vertraulichen Daten und Informationen der \varUnternehmen. 
In diese Arbeit dürfen Dritte, mit Ausnahme der Gutachter und befugten Mitgliedern des Prüfungsausschusses, 
ohne ausdrückliche Zustimmung des Unternehmens und des Verfassers keine Einsicht nehmen. Eine Vervielfältigung 
und Veröffentlichung der \varArbeit \space ohne ausdrückliche Genehmigung – auch auszugsweise – ist nicht erlaubt.
    \setstretch{1} 
    \pagebreak
  \fi

  % Seitenzahlen auf große römische Zahlen umstellen
  \pagenumbering{Roman}

  % Inhaltsverzeichnis
  \tableofcontents
  \newpage
  
  % Abbildungsverzeichnis
  \iftotalfigures
    \listoffigures
  \fi

  % Tabellenverzeichnis
  \iftotaltables
    \listoftables
  \fi

  % Listingverzeichnis
  \iftotallstlistings
    \addcontentsline{toc}{section}{Listingverzeichnis}
    \lstlistoflistings
  \fi
  \newpage

  % Abkürzungsverzeichnis
  \addcontentsline{toc}{section}{Abkürzungsverzeichnis}
  \setstretch{0.5} 
  \printnoidxglossary[type=acronym,sort=letter,style=listdotted,title=Abkürzungsverzeichnis]
  \newpage

  % Arabische Seitennummerierung für den Hauptteil
  \pagenumbering{arabic}

  % Um den Zeilenabstand der Wordvorlage anzupassen
  \setstretch{1.2}
  
  % Inhalt der Arbeit -> Strukturierung in section/root
  \section{Demo Section}
  \label{demo}  
  \section{Demo Section}\label{demo}  

Dieser Abschnitt soll eine kleine Demo einzelner Funktionalitäten darstellen, indem die jeweiligen Beispiele in einem Kontext vorgestellt werden.
\footcite[An dieser Stelle soll erneut auf die Lizenz hingewiesen werden. Siehe][]{ccOJlizenz}
Für Grundlegendes kann sowohl die README-Datei als auch das Latex-Sheet zu Rate gezogen werden. 
Über die einzelnen Themen sind Verwendungen von Abkürzungen und Fußnoten wie dieser\footrefnote{Lipsum}{lipsum} verteilt. 

Zeilenumbrüche in der PDF-Datei können entweder mittels zwei Zeilenumbrüche in der Quelldatei erzeugt werden, wie vor dieser Zeile\\
oder durch doppelten Backslash wie vor dieser Zeile.

\subsection{Fließelemente}
  Diese umfassen Bilder, Tabellen sowie Listings.
  Während sich das Beispiel für Listings im Anhang~\ref{appendix_dummy} befindet, zeigt dieses Kapitel die Nutzung der anderen Elemente auf.

 \subsubsection{Bilder}
    \begin{wrapfigure}{r}{0.45\linewidth}
      \vspace{-30pt}
      \centering
      \includegraphics[width=0.45\textwidth]{mvc}
      % caption[im Verzeichnis]{unter der Abbildung}
      \caption[Aufbau von \acrlong{MVC}]{Aufbau von \gls{MVC}.\\Quelle: Angelehnt an \cite{curry2008flexible}}
      \label{fig_mvc}
    \end{wrapfigure}
    Bilder können sowohl als eigenständiges Blockelement, als auch mit dem Text in einer Zeile mit fließen.
    Jedoch entscheidet \LaTeX, wo genau die Bilder angezeigt werden, weshalb sie nicht unbedingt an der Stelle erscheinen,
    an der sie definiert wurden.
    Jedoch können (und sollten) Bilder im Text über das vergebene Label referenziert werden:
    Durch diese Verbindungen entsteht eine Dreiecksform, welche in Abbildung~\ref{fig_mvc} illustriert wird.

    \begin{figure}[tbh]
      \centering
      \includegraphics[width=\textwidth]{vsc_latex_workshop}
      \caption{Einblick in die Entwicklungsumgebung Visual Studio Code}
      \label{fig_prototyp_desktop}
    \end{figure}

\subsubsection{Tabellen}
  Tabellen funktionieren wie die erste Art von Bildern; als fließendes Blockelement.

  \begin{table}[tb]
    \centering
    \begin{tabular}{@{}lcccccc@{}}
    \toprule
    Abteilung               & Marketing & Vertrieb & Produktion & IT \\ \midrule
    Besetzte Stellen        & 11        & 16       & 31         & 4  \\
    Unbesetzte Stellen      & 1         & 2        & 0          & 2  \\
    Neu geschaffene Stellen & 3         & 1        & 0          & 2  \\
    Geplante neue Stellen   & 2         & 0        & 2          & 0  \\ \bottomrule
    \end{tabular}
    \caption{Übersicht besetzter und geplanter Stellen im Unternehmen}
    \label{tbl_proglang}
  \end{table}

\subsection{Zitate}
  In \LaTeX gibt es verschiedene Arten wörtlich zu zitieren.
  Zum einen kann es inline durch einfache Anführungszeichen gemacht werden, 
  zum anderen gibt es eine fertige Umgebung, welche das Zitat im Dokument besonders hervorhebt.
  
  \subsubsection{Inline Zitat}
    "`\gls{AWATAD}"'\footcite[S. 2703]{timmerer2019journey}, zu deutsch überall, jederzeit, auf jedem Gerät,
    sind die heutigen Erwartungen der Nutzer im Kontext von Webapplikationen.
    \vgl{timmerer2019journey}{S. 2703--2704}    
    Hierbei ist auf die Codierung der Anführungszeichen zu achten.
  
  \subsubsection{Blockzitat}
    Balzert definiert Softwaretechnik als
    \begin{quotation}
      "`Zielorientierte Bereitstellung und systematische Verwendung von Prinzipien, Methoden und Werkzeugen 
      für die arbeitsteilige, ingenieurmäßige Entwicklung und Anwendung von umfangreichen Softwaresystemen."'\footcite[S. 17]{balzert2009lehrbuch}    
    \end{quotation}
    Dabei betont er, dass etablierte Prinzipien und Methoden, sogenannte 
    \emph{Best Practices}\footnote{\emph{engl.} empfohlene Vorgehensweise},
    an Stelle von eigenen erfundenen Mustern zu verwenden sind.
    Des Weiteren grenzt er Technologien wie Betriebssysteme, Netzwerke und Datenbanken als Randbereiche ab. 
    Softwareentwicklung ist also die Gesamtheit aller Schritte von der Anforderungsanalyse bis hin zur Wartung eines Softwaresystems
    und lässt sich in die Teildisziplinen \emph{Softwareentwicklung}, \emph{Softwaremanagement} und \emph{Softwarequalitätsmanagement} untergliedern.
    \wholesection{balzert2009lehrbuch}{S. 17--19}

\subsection{Textstrukturierung}
  \subsubsection{Gliederungsebenen}
    In der verwendeten Article-Dokumentenklasse werden standardmäßig drei Hierarchieebenen unterstützt.
    Diese werden in folgender Aufzählungsumgebung kurz erläutert.
    
    \begin{itemize}
      \item sections stellen die oberste Ebene dar und werden in dieser Demo in der Datei root.tex definiert.
      \item subsection eine Ebene darunter
      \item darunter folgt subsubsection, wofür dieses Kapitel ein Beispiel ist 
    \end{itemize}

    \paragraphheader{Bonusebene}
    Falls - wie hier - mehr Gliederungsebenen benötigt werden, als die Dokumentklasse zulässt, 
    kann mittels dem vordefinierten paragraphheader-Befehl eine weitere, unnummerierte Ebene eingefügt werden.
   
  \subsubsection{Mathe-Modus}
    Der Mathe-Modus, der bekanntlich für Formeln genutzt wird, lässt sich durch \$ einfach anstellen:
    $E = m \cdot c^2$
\conditionalPageBreak

\section{Lipsum}
  \label{lipsum}
  In diesem Kapitel soll Bildtext etwas Platz einnehmen,
damit die entstehende PDF etwas mehr das Aussehen 
einer fertigen Arbeit annimmt.

\subsection{Unterkapitel im Lipsum-Kapitel}
  \lipsum[1-2]
  \subsubsection{Ein Kapitel auf dritter Ebene}
    \lipsum[3-5]
  \subsubsection{Ein weiteres Kapitel}
    \lipsum[4-6]

\subsection{Zweites Unterkapitel im Lipsum-Kapitel}
  \lipsum[7-8]
  \subsubsection{Und noch ein Kapitel}
    \lipsum[9-10]
  \subsubsection{Das letzte Kapitel}
    \lipsum[11]
\clearpage

  \newpage

  % Literaturverzeichnis
  \setstretch{1.1} 
  \sloppy
  \hbadness=2000
  \printbibliography[heading=bibintoc]
  \newpage

  % Persönliche Erklärung
  \pagenumbering{gobble}
  \section*{Persönliche Erklärung}

Hiermit erkläre ich, dass ich
\begin{enumerate}
  \item meine \varArbeit \space ohne fremde Hilfe angefertigt habe,
  \item die Übernahme wörtlicher Zitate aus der Literatur sowie die Verwendung von Gedanken anderer Autoren an den entsprechenden Stellen innerhalb der Arbeit gekennzeichnet habe und
  \item meine \varArbeit \space bei keiner anderen Prüfungsstelle vorgelegt habe.
\end{enumerate}
Ich bin mir bewusst, dass eine falsche Erklärung zum Nichtbestehen der \varArbeit \space führt.

\vspace{2cm}

\begin{tabular}{lp{2em}l}
 \hspace{5cm}   && \hspace{3cm} \\\cline{1-1}\cline{3-3}
 Ort, Datum     && Unterschrift
\end{tabular}


\end{document}